\documentclass[fleqn,a4paper,nobib]{tufte-handout}

\usepackage[]{amsmath,amsthm,amssymb}
\usepackage[]{systeme}
% accès à enumerate
\usepackage{enumitem}

%Set the font (output) encoding
%--------------------------------------
\usepackage[T1]{fontenc} %Not needed by LuaLaTeX or XeLaTeX

%French-specific commands
%--------------------------------------
% nécessite fontenc (voir au dessus)
\usepackage[french]{babel}
\usepackage[autolanguage]{numprint} % for the \nombre command

%Hyphenation rules
%--------------------------------------
\usepackage{hyphenat}
\hyphenation{mate-mática recu-perar}

\usepackage{xcolor,graphicx}

\newcommand{\colornote}[2][gray!40]{
    \hphantom{hello}

    %\noindent
    \hspace{-\parindent}
    \hspace{-12pt}
    \colorbox{#1}{
        \begin{minipage}{\textwidth}
            %\vspace{4pt}
            \parindent12pt
            #2
            \vspace{4pt}
        \end{minipage}
    }

    \hphantom{hello}
}

\newcommand{\margincolornote}[2][gray!20]{
    \noindent
    \colorbox{#1}{
        \begin{minipage}{\textwidth}
            %\vspace{4pt}
            #2
        \end{minipage}
    }
}

\newcommand{\greybox}[2][gray!40]{
    \colornote[#1]{#2}
}

\newcommand{\margingreybox}[2][gray!20]{
    \margincolornote[#1]{#2}
}

\newcommand{\textmath}[1]{
    \textit{#1}
}

\usepackage{csquotes}

\newcommand{\syllogism}[3]{
    \begin{flalign*}
        &\textit{#1}&\\
        &\textit{#2}&\\
        \qquad \qquad \therefore \hphantom{z} &\textit{#3}&
    \end{flalign*}
}

% Quelques classes de théorèmes
\newtheorem{definition}{Définition}
\newtheorem{theorem}{Théorème}
\newtheorem{lemma}{Lemma}[theorem]
\newtheorem{corollary}{Corollaire}[theorem]
\newtheorem*{remark}{\normalfont{\emph{Remarque}}}
\newtheorem{exercise}{\normalfont{\emph{Exercice}}}
\newtheorem*{solution}{\normalfont{\emph{Solution}}}

\usepackage[]{float}

\usepackage{array}
\usepackage{siunitx}

\usepackage{booktabs}

\usepackage[]{basicarith}


% les plots tikzpicture
\usepackage{pgfplots}
\pgfplotsset{compat = newest}
\usetikzlibrary{intersections}

% margin plot
\newenvironment{marplot}[2]{
    \begin{marginfigure}[]
        \caption{\footnotesize #2}
        \begin{tikzpicture}[]
            \begin{axis}[#1]
}{
            \end{axis}
        \end{tikzpicture}
    \end{marginfigure}
}

\newcommand{\colorsquare}[1][red]{
    $\color{#1}\blacksquare$
}

\title{Logique formelle}
\author{Tanguy Andreani}
\date{Dernière mise à jour: \today}

\begin{document}

\maketitle

\begin{abstract}
Notes prises pour les livres
\underline{An Introduction to Formal Logic} (Smith)
et \underline{forall $x$: Calgary} (Button, Magnus, Loftis, Trueman, Thomas-Bolduc, Zach).

D'autres sources s'ajoutent: les autres écrits de Peter Smith et le site
\url{https://plato.stanford.edu/}.
\end{abstract}

\tableofcontents

\section{Introduction}

On va développer tout au long de ces notes une vision de plus
en plus formalisée de la logique.

On commence par créer une base informelle pour créer et identifier
des arguments à partir de connecteurs du langage courant et identifier
des propriété ou des relations entre les composantes d'un argument.\\

\textbf{Attention. Ce texte est certainement rempli d'erreurs.}


\section{Développer un langage}

Cette section est plus orientée \textit{problem solving}
et sert surtout à introduire du vocabulaire et des grandes idées.

Il est difficile de cerner exactement ce que couvre la logique
formelle. On développe un langage dédié qui permettra d'exprimer
sans ambiguïté ce que Calgary nomme "l'état des choses". Si on tient
les faits $A$, $B$ et $C$ pour vrais, que peut on en déduire, est-il
possible de dériver de nouveau faits à partir de cette certitude ? Les
"phrases" de ce langage peuvent être liées entre elles par
plusieurs types de relations. Par exemple,

\begin{description}
    \item[L'implication] le fait que certaines choses soient si certaines autres sont.
    \item[La contradiction] le fait que certaines choses ne puissent pas être simultanément.
\end{description}

Calgary distingue les définitions \emph{sémantiques} de ces
relations de celles basées sur une théorie des preuves.
\sidenote{
    Dans le premier cas c'est l'interprétation du langage et
    dans le second l'utilisation d'un système formalisé de déduction.
    J'ai tout de même l'impression que les deux s'équivalent.
}

La structure que l'on va étudier est celle de l'\textit{argument}.

\begin{description}
    \item[Un argument] est un raisonnement qui à vocation à entraîner une conclusion.
    Il se compose de prémisses qui mènent, par une ou plusieurs étapes d'inférence, à
    la conclusion.
    \item[Un prémisse] est une sorte de fait initial sur lequel on se base pour
    créer de nouveaux faits.
    \item[Une conclusion] a la même forme qu'un prémisse et est le point d'arrivée
    d'un argument, au moyen d'un \textbf{raisonnement déductif}.
\end{description}

\subsection{Le raisonnement déductif}

Le raisonnement est quelque chose que l'on va tenter de formaliser. C'est
le processus mental qui permet par exemple, de dire que\textmath{si je suis un
humain et que tous les humains sont mortels, alors je suis mortel}.

Le mot \textit{alors} marque la conclusion de l'argument, mais comment être sûr
que cette conslusion découle bien des prémisses?

Un raisonnement, ou étape d'inférence, peut être valide ou invalide.

Pour évaluer la validité d'un raisonnement déductif, Smith introduit l'idée de
\textbf{cohérence interne} des arguments. On évalue un argument sans prendre
en compte les paramètres extérieurs (comme la validité \emph{réelle} des prémisses
par exemple).

La validité des prémisses et celle du raisonnement déductif (l'ensemble
des \textit{inference steps})
sont deux étapes distinctes de l'évaluation d'un argument. Ainsi,
un raisonnement peut être parfaitement valide bien que basé sur des prémisses
%faux. (Ex:\textmath{Si je suis un lapin et que tous les lapins tirent des lasers, alors
%je tire des lasers.})

\greybox{
    \subsection*{L'usage des mots <<inférence>> et <<déduction>>}

    \begin{displayquote}
        En logique, l’inférence est un mouvement de la pensée allant des
        principes à la conclusion. (\href{https://fr.wikipedia.org/wiki/Inférence}{source: Wikipedia/Inférence})
    \end{displayquote}

    En Français, le mot déduction se rapporte au raisonnement déductif.
    Celui-ci désigne un type d'inférence.

    \begin{displayquote}
        La déduction est une opération par laquelle on établit au moyen
        de prémisses une conclusion qui en est la conséquence nécessaire,
        en vertu de règles d'inférence logiques.
    \end{displayquote}
    
    Une inférence \textit{inductive}, c'est quand la conclusion est
    probable ou très probable. Par exemple le raisonnement d'un détective
    est inductif. Elle est incompatible avec l'inférence déductive.
}

\subsection{Généraliser des formes d'arguments par l'analyse sémantique}

Plutôt que d'évaluer au cas par cas des arguments écrits sous forme
prosaïque, on peut identifier des structures qui se répètent et
vérifier si les arguments en question y correspondent.

On peut faire ça car on ne se souci que de la cohérence \textit{interne}
des arguments. Ainsi, la nature exacte des choses dont l'argument
traite n'a pas d'importance.

Un argument peut être \textbf{nomologiquement valide} si il ne lui
existe aucun contre exemple qui respecte les lois de la nature.
Il est \textbf{conceptuellement valide} si on peut démontrer sa
cohérence interne. Un argument est dit \textit{solide} si il est
valide et que ses prémisses sont vrais. En logique, vu qu'on ne
s'intéresse qu'à la validité de l'argument, qu'un argument soit
nomologiquement valide importe peu.


Tout argument de la forme
\sidenote{C'est un schéma de déduction, on revient dessus en section 2.}
\syllogism{Tous les F sont G}{n est F}{n est G}
est valide dans ce cadre, mais pas forcément nomologiquement.

Cette façon de procéder correspond plus à une méthode d'évaluation
des arguments basée sur la sémantique. Certains mots sont de bons
indicateurs pour passer identifier les prémisses et la conclusion

Cette façon de procéder correspond plus à une méthode d'évaluation
des arguments basée sur la sémantique. Certains mots sont de bons
indicateurs pour passer identifier les prémisses et la conclusion
dans un argument sous forme de prose.
\begin{marginfigure}[-5cm]
    \margingreybox{
        \textbf{indicateurs de conclusion}\\
        so, therefore, hence, thus, accordingly, consequently [\textit{conclusion}]\\
        Aussi appelés \textit{inference markers}.\\
        \textbf{indicateurs de prémisses}\\
        since, because, given that [\textit{premise}]
    }
\end{marginfigure}

\subsection*{Exercices (Calgary p. 17)}

Essayer d'identifier si des arguments sont valides ou non en
regardant leur forme.

\begin{enumerate}[label=\textbf{\Alph*.}]
    \item Arguments valides:
    \syllogism{Socrates is a man.}{All men are carrots.}{Socrates is a carrot.}
    \syllogism{Abe Lincoln was either born in Illinois or he was once president.}{Abe Lincoln was never president.}{Abe Lincoln was born in Illinois.}
    \syllogism{Abe Lincoln was either from France or from Luxembourg.}{Abe Lincoln was not from Luxembourg.}{Abe Lincoln was from France.}
    \syllogism{If the world ends today, I will not need to get up tomorrow.}{I will need to get up tomorrow.}{The world will not end today.}

    Arguments invalides:
    \syllogism{If I pull the trigger, Abe Lincoln will die.}{I do not pull the trigger.}{Abe Lincoln will not die.}

    Abe Lincoln pourrait mourir d'autre chose.
    \syllogism{Joe is now 19 years old.}{Joe is now 87 years old.}{Bob is now 20 years old.}
    Pour le dernier argument, on a un argument de la forme
    \syllogism{A est F}{A est G}{B est H}

    Sémantiquement, on voit bien qu'en l'absence de référence à $B$ et $H$
    dans les prémisses, on ne peut pas évaluer leur impact sur la conclusion
    sans faire référence à d'éléments extérieurs. Les deux prémisses
    sont incompatibles dans notre perception des choses ce qui fait que
    notre argument ne serait même pas nomologiquement valide.

    \item (Uniquement mes réponses)
    
    \begin{enumerate}[label=\arabic*.]
        \item Un argument valide peut très bien être défaillant dans
        certains de ses prémisses.
        \item Un argument peut être défaillant dans tous ses prémisses.
        \item Un argument valide peut aussi avoir une conclusion fausse, tout
        ce qui nous intéresse c'est le raisonnement déductif.
        \item Un argument invalide pourrait très bien devenir valide par
        l'ajout d'un prémisse. Le raisonnement change.
        \syllogism{Je ne suis pas grand.}{Je ne suis pas petit.}{Je suis de taille moyenne.}
    \end{enumerate}
\end{enumerate}

\subsection{La validité des propositions}

\begin{description}
    \item[Une proposition] est un candidat acceptable pour être un prémisse ou
    une conclusion dans un argument.
\end{description}

On ne peut pas la définir dans l'absolu mais on peut admettre
que certaines propositions soient vraies.


Considérées sous le prisme d'une \textit{situation}, les propositions
peuvent être reliées entre elles.

\begin{enumerate}[label=\arabic*.]
    \item Plusieurs propositions sont \textit{inconsistentes} si et seulement si
    il n'y a aucune situation possible dans laquelle ses propositions
    sont toutes vraies simultanément.
    \item Plusieurs propositions sont \textit{équivalentes} si et seulement si
    elles sont vraies dans exactement les mêmes cas. 
\end{enumerate}

\subsection{La validité de l'inférence}

\begin{displayquote}
    Un argument est valide si et seulement si\sidenote{\footnotesize\textit{si et seulement si} est utilisé ici dans le sens où il est impossible qu'un argument soit valide si il existe une telle situation. Du coup c'est méta.} il n'existe aucune situation
    dans laquelle les prémisses seraient vrais et la conclusion fausse.
\end{displayquote}

La notion de situation est un peu floue. 
On en revient à cette notion de coherence \textit{interne} de l'argument.
La logique s'inscrit dans un "univers" de pensée qui n'est pas limité
par les éléments exterieurs que nous connaissont. Dans les exemples
précédents, cet univers est contenu dans la formulation. Restent des
axiomes qui permettent de construire des schémas de pensée valides.
Par exemple, le fait de ne pas pouvoir dire que \textmath{A est C}
et \textmath{A n'est pas C} en même temps.

La notion de \textit{possibilité} ou de probabilité d'une situation
doit être prise au sens le plus large et le plus inclusif. Soit
quelque chose est possible soit il ne l'est pas, et si la probabilité
entre les deux est infime, alors il est possible.

\greybox{
\subsection*{La possibilité qu'un prémisse ou qu'une conclusion soit vrai.}

Vu qu'on ne sait pas très bien à ce stade ce qu'implique cette
définition de ce qui arrive "probablement", on pourrait utiliser un autre mot,
sans racine: \textit{M-ment} pour désigner son sens le plus large.

\begin{displayquote}
    Il est \textit{M}-ment nécessaire que C si et seulement si il n'est pas \textit{M}-ment possible
    que $\bar{C}$.
\end{displayquote}

L'inférence, si elle est valide,
préserve la \textit{vérité}. Passer des prémisses à la conclusion
conserve la vérité. \textit{On passe de "vrai à vrai" mais pas forcément de
"faux à faux".}
}

\subsection*{Le principe d'invalidité}

Le \textit{principe d'invalidité} dit que si un argument comprend
uniquement des prémisses vrais et une conclusion fausse
alors il est forcément invalide.

Ça pose une difficulté. Si on ne prend pas en compte les
éléments exterieurs comme la validité des prémisses, de quelle
vérité parle-t-on ici?

De celle qu'on admet comme base pour
l'exercice.

\greybox{
    \subsection*{Les arguments solides}

    À partir du moment où l'on peut être $M$-ment certains que les prémisses
soient vrais, on peut tirer des conclusions de l'existence d'arguments
solides par opposition à ceux qui tombent sous le principe d'invalidité.

\begin{enumerate}
    \item Tout argument solide a une conclusion vraie.
    \item Deux arguments solides ne peuvent avoir de conclusions inconsistentes entre elles.
    \item Aucun argument solide n'a de prémisses inconsistentes entre elles.
\end{enumerate}
}

\subsection*{Exercices (Smith p. 19)}

\begin{enumerate}[label=(\alph*)]
    \item (Vrai ou Faux)
    \begin{enumerate}[label=(\arabic*)]
        \item \textit{Les prémisses et la conclusion d'un argument invalide sont
        incompatibles.} \textbf{Faux.} En effet, un argument dont les prémisses
        et la conclusion sont vraies peut être invalide par une étape d'inférence
        défaillante.
        \item \textit{Si un argument a des prémisses fausses et une conclusion vraie,
        alors la validité de la conclusion ne peut pas être attribuée à
        celle des prémisses, donc l'argument ne peut être valide.} \textbf{Faux.}
        Dans le cadre de notre système, la validité d'une étape d'inférence
        ne dépend pas de la validité des prémisses et de la conclusion, sauf
        pour invalider un argument au moyen du principe d'invalidité, ce
        qui n'est pas le cas ici.
        \item \textit{Tout étape d'inférence qui implique des prémisses vraies et une
        conclusion vraie préserve la vérité et est donc valide.} \textbf{Faux.}
        On pourrait dire que l'inférence préserve la vérité si elle est valide
        sans prendre en compte la validité des prémisses.
        \item \textit{On peut rendre un argument valide invalide en ajoutant des prémisses.}
        \textbf{Vrai.} Il suffit d'ajouter comme prémisse la négation de la conclusion.
        (Si on ajoute la négation de la compatibilité entre les prémisses initiaux,
        ça rend l'argument invalide ?)
        \item \textit{On peut rendre un argument solide non solide en ajoutant des prémisses.}
        \textbf{Vrai.} Il suffit de le rendre invalide ou de rajouter un prémisse faux.
        \item \textit{On peut rendre un argument invalide valide en ajoutant des prémisses.}
        \textbf{Vrai.} Si on a un argument de la forme \textit{A est I, J ou K. A n'est pas I. Donc A est K.}
        On peut le rendre valide en ajoutant \textit{A n'est pas J.}
        \item \textit{Si des propositions sont consistentes entre elles, ajouter une
        proposition vraie ne peut les rendre inconsistentes.} \textbf{Faux.}
        Si on a des propositions de la forme \textit{A est I. A est J.} On peut les
        rendre incompatible avec l'introduction de \textit{A est soit I soit J.} comme
        vraie.
        \item \textit{Si des propositions sont inconsistentes entre elles, alors
        on ne peut les rendre consistentes par l'ajout d'un nouveau prémisse.}
        \textbf{Vrai.} Avec l'exemple précédent, si on introduit \textit{A peut être à
        la fois I et J}, il y a quand même incompatibilité entre \textit{A est I soit J}
        et \textit{A est I} et \textit{A est J}.
        \item \textit{Si des propositions sont compatibles, leur négation le sont aussi.}
        \textit{Faux.} \textit{A est I. A est J. A est soit I soit J.} Les négations des
        deux premières propositions sont incompatibles.\sidenote{\footnotesize Théorème de De Morgan?}
        \item \textit{Si des propositions sont inconsistentes entre elles, alors on peut
        sélectionner n'importe laquelle et argumenter qu'elle est invalide si les autres
        propositions sont vraies.} \textbf{Faux.} Les autres propositions sont peut être
        incompatibles entre elles.
    \end{enumerate}
    \item[(b*)] (Juste mes démonstrations. Cet exercice introduit une méthode qui s'éloigne
    un peu de la simple analyse sémantique d'un argument.)
    \begin{enumerate}[label=(\arabic*)]
        \item Si $A$ implique $C$ et $C$ est équivalent de $C'$ il n'y a que deux cas possibles: \\
            \begin{tabular}{|c|c|c|}
                \hline
                $A$ & $C$ & $C'$ \\
                \hline
                V & V & V \\
                F & ? & ? \\
                \hline
            \end{tabular} \\
            \textit{Où le point d'interrogation dénote la validité de $C$ lorsque $A$ est faux.}

            On voit bien que la relation entre $A$ et $C'$ est la même
            que celle entre $A$ et $C$ et que cette relation conserve
            la validité ce qui est la définition de l'implication.
            On a donc $A$ implique $C'$.

            Je ne sais pas si c'est la bonne manière de démontrer. En tout cas
            lister les cas possibles me semble plus sain que de paraphraser
            l'énoncer.
        \item Si $A$ implique $C$ et $A$ est équivalent de $A'$ alors il n'y a que
        deux cas possibles: \\
        \begin{tabular}{|c|c|c|}
            \hline
            $A$ & $A'$ & $C$ \\
            \hline
            V & V & V \\
            F & F & ? \\
            \hline
        \end{tabular} \\
        La relation entre $A'$ et $C$ conserve la validité, donc $A'$ implique $C$.
        \item Si $A$\textmath{et\sidenote{\footnotesize $A$\textmath{et}$B$ est une
        propositions vraie uniquement si $A$ et $B$ sont vrais.}}$B$ implique $C$,
        et $A$ est équivalent de 
        $A'$, il y a 4 cas possibles: \\
        \begin{tabular}{|c|c|c|c|}
            \hline
            $A$ & $A'$ & $B$ & $C$ \\
            \hline
            V & V & V & V \\
            V & V & F & ? \\
            F & F & V & ? \\
            F & F & F & ? \\
            \hline
        \end{tabular} \\
        La relation entre $A$\textmath{et}$B$ et $C$ est la même que
        celle entre $A'$\textmath{et}$B$ et $C$. Cette relation conserve
        la validité.
    \end{enumerate}

    On peut dire que des propositions équivalentes se comportent
    de la même manière dans les arguments qui les contiennent.
    Deux propositions équivalentes sont vraies dans exactement les
    même <<cas>>.

    Autrement dit, il est impossible de formuler une situation cohérente où l'une
    est vraie et pas l'autre. À partir de là, les deux propositions
    sont indistinguables du point de vue de leur vérité.

    Un argument peut être valide ou invalide en fonction de l'inférence
    ou du principe d'invalidité. Dans un argument qui tombe sous le
    principe d'invalidité, remplacer un prémisse par un prémisse
    équivalent ne change rien à la validité de l'argument, puisqu'on
    ne regarde que la validité du prémisse.
\end{enumerate}

\subsection{Une définition circulaire}

Concrètement, il est difficile de définir la compatibilité entre deux
propositions. Par exemple, quelles étapes doivent être franchies
pour s'assurer
de cette compatibilité?

À la fin de notre travail, on ne doit pas pouvoir déduire de
contradiction avec ce qu'on a produit. En tout cas Smith écrit
qu'à ce stade, on pourrait penser que c'est la condition
minimale à tout raisonnement valide. Seulement, c'est une définition
circulaire (de la logique qu'on essaye de formaliser en fonction
d'elle même).

L'idée même que la validité d'une proposition soit cohérente au
sens \textit{M-ment possible} est troublante puisque pour
vérifier cette cohérence, cette possibilité, il faut un autre
argument.

\section{Les schémas de déduction}
\marginnote{(Le titre de cette section est peut être mal traduit.)}

\greybox{
    \subsection*{Les schémas pour représenter des formes de déduction}

    Les schémas sont un peu comme des fonctions.
    \syllogism{No F is G}{n is F}{n is not G}

    $F$, $G$ et $n$ sont des \textit{variables} du schéma.
    \syllogism{No F is a man}{n is F}{n is not a man}

    est aussi un schéma.

    On dit qu'un schéma dont toutes les \textit{instances}
    sont valides est lui même valide.

    On peut associer à un argument plusieurs formes de déduction,
    on choisi typiquement celle qui est la plus générale. (Dans cet
    encadré, le premier exemple est plus général que le second.)
}

\subsection*{Exercices (Smith p. 26)}

\begin{enumerate}[label=(\alph*)]
    \item Quelles formes d'inférence sont valides ?
        \begin{enumerate}[label=(\arabic*)]
            \item \textmath{Certains F sont G; aucun G n'est H; donc, certains F ne sont pas H.}
            \textbf{Valide.}
            \item \textmath{Certains F sont G; certains H sont F; donc, certains G sont H.}
            \textbf{Invalide.} \textmath{H = déménageurs, F = des personnes fortes, G = haltérophiles}.
            De tout évidence, il est très possible qu'aucun déménageur ne soit aussi haltérophile.
            \item \textmath{Tous les F sont G; certains F sont H; donc, certains H sont G.}
            \textbf{Valide.} Certains $F$ sont $G$ et $H$. Donc certains $H$ sont $G$.
            \item \textmath{Aucun F n'est G; certains G sont H; donc, certains H ne sont pas F.}
            \textbf{Invalide.} Dire que \textmath{Tous les H sont F} n'est pas incompatible avec
            les prémisses mais l'est avec la conclusion.
            \item \textmath{Aucun F n'est G; aucun H n'est G; donc, certains F ne sont pas H.}
            \textbf{Invalide.} Dire que \textmath{Tous les F sont H} n'est pas incompatible
            avec les prémisses mais l'est avec la conclusion.
            \item \textmath{Tous les F sont G; aucun G n'est H; donc, aucun H n'est F.}
            \textbf{Valide.} Aucun $F$ n'est $H$ donc aucun $H$ n'est $F$.

        \end{enumerate}
    \item Quelles formes d'inférence sont valides ? (Formes un peu plus exotiques.)
        \begin{enumerate}[label=(\arabic*)]
            \item \textmath{Tous les F sont G; donc, rien de ce qui n'est pas G n'est F.}
            \textbf{Valide.} Il est impossible possible de concevoir un $n$ qui serait $F$ mais pas $G$.
            \item \textmath{Tous les F sont G; aucun G n'est H; certains J sont H; donc, certains J ne sont pas F.}
            \textbf{Valide.} Aucun $J$ qui est à la fois $H$ n'est $F$, donc, on a bien \textmath{certains J ne sont pas F.}
            \item \textmath{Il y a un nombre impair de F, il y a un nombre impair de G; donc il y a
            un nombre pair de choses qui sont soit G soit F.} \textbf{Invalide.} Si on a 3 éléments qui sont $F$,
            et 3 éléments qui sont $G$ mais que un élément est $F$ et $G$, alors il y a en tout 5 éléments.
            \item \textmath{Tous les F sont G; donc, au moins une chose est F et G} \textbf{Invalide.}
            Il peut très bien n'exister aucun $F$, par contre si il existe des $F$, ils sont forcément $F$ et $G$.
            \item \textmath{m est F; n est F; donc il y a au moins deux F} \textbf{Valide.}
            Ici, la formulation implique que $m$ et $n$ existent bel et bien. Dans l'exemple précédent,
            on raisonnait avec des propriétés ($F$, $G$...) et non avec des objets particuliers.
            Je note qu'on a pas encore bien défini se qui constitue un objet ou une propriété d'un objet
            en logique.
            \item \textmath{Tout F est G; aucun G n'est H; donc, tout J est J.} \textbf{Valide mais.}
            La conclusion est forcément vraie, donc le raisonnement ne tombe pas sous le principe
            d'invalidité.
        \end{enumerate}
\end{enumerate}

L'exerice (a) ne traite que de \textit{syllogismes}. Un type d'argument très simple
décris en premier par Aristote.

\greybox{
    \subsection*{Les syllogismes}
    
    Un syllogisme se compose de trois propositions.
    Chaque proposition peut prendre une des formes suivantes
    :
    
    \begin{enumerate}[leftmargin=3cm]
        \item[A:] Tous les X sont Y
        \item[E:] Aucun X n'est Y
        \item[I:] Certains X sont Y
        \item[O:] Certains X ne sont pas Y   
    \end{enumerate}
    
    Les deux termes qui apparaissent dans la conclusion "viennent" chacun d'un
    prémisse différent. Et on rajoute un terme "commun" aux deux prémisses
    mais absent de la conclusion.
}
\marginpar{\vspace{-3.9cm}(On suppose que ces labels viennent des mots
latins \emph{affirmo} et \emph{nego}.)}

\begin{enumerate}[label=(\alph*)]
    \item[(c)] Quels syllogismes peuvent avoir une conclusion de forme A ? Pareil pour O.
        \begin{enumerate}[label=(\arabic*)]
            \item Les formes de syllogismes qui ont une conclusion de la forme A:
            \syllogism{Tous les S sont M}{Tous les M sont P}{Tous les S sont P}
            \item Les formes de syllogismes qui ont une conclusion de la forme O:
            \syllogism{Certains S sont M}{Aucun S n'est P}{Certains S ne sont pas P}
        \end{enumerate}
        Je ne vois pas très bien comment énumérer toutes les possibilités.
    \item[(d)] Les stoïciens se concentrait sur une autre famille d'argument, qu'ils
    considéraient indémontrables par leur simplicité. Il utilisait une syntax où $A$ et
    $B$ désignent des propositions et le préfixe \textit{non-} leur négation. 
        \begin{enumerate}[label=(\arabic*)]
            \item \textmath{Si A alors B; A; donc B} Semble valide par la définition de l'implication.
            \item \textmath{Si A alors B; non-B; donc non-A} Semble valide par la définition de l'implication.
            Je note que c'est la \textit{contraposée} de l'implication qui est utilisée ici. Si $A$ implique $B$ alors
            $non-B$ implique $non-A$.
            \item \textmath{non-(A et B); A; donc non-B} Valide car si $B$ était vrai, alors \textmath{(A et B)}
            serait vrai ce qui est une contradiction.
            \item \textmath{A ou B; A; donc non-B} Invalide car $A$ et $B$ peuvent être valides simultanément.
            \item \textmath{A ou B; non-A; donc B} Valide car $A$ et $B$ ne peuvent être faux en même temps.
            \item \textmath{Si A alors B; non-A; donc non-B} Invalide car $B$ pourrait être vrai indépendamment de $A$.
            \item \textmath{Si A alors B; B; donc A} Invalide car $B$ pour la même raison.
            \item \textmath{non-(A et B); donc soit non-A soit non-B} Invalide car $A$ et $B$ peuvent être faux en même temps.
            \item \textmath{A or B; donc non-(non-A et non-B)} Valide car $A$ ne peut être faux en même temps que $B$.
            D'ailleurs c'est le théorème de DeMorgan.
            \item \textmath{non-non-A; donc A} Si $A$ est faux alors \textmath{non-A} est vrai ce qui est une contradiction.
        \end{enumerate}
\end{enumerate}



\end{document}