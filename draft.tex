\documentclass[fleqn,a4paper,nobib]{tufte-handout}

\usepackage[]{amsmath,amsthm,amssymb}
\usepackage[]{systeme}
% accès à enumerate
\usepackage{enumitem}

%Set the font (output) encoding
%--------------------------------------
\usepackage[T1]{fontenc} %Not needed by LuaLaTeX or XeLaTeX

%French-specific commands
%--------------------------------------
% nécessite fontenc (voir au dessus)
\usepackage[french]{babel}
\usepackage[autolanguage]{numprint} % for the \nombre command

%Hyphenation rules
%--------------------------------------
\usepackage{hyphenat}
\hyphenation{mate-mática recu-perar}

\usepackage{xcolor,graphicx}

\newcommand{\colornote}[2][gray!40]{
    \hphantom{hello}

    %\noindent
    \hspace{-\parindent}
    \hspace{-12pt}
    \colorbox{#1}{
        \begin{minipage}{\textwidth}
            %\vspace{4pt}
            \parindent12pt
            #2
            \vspace{4pt}
        \end{minipage}
    }

    \hphantom{hello}
}

\newcommand{\margincolornote}[2][gray!20]{
    \noindent
    \colorbox{#1}{
        \begin{minipage}{\textwidth}
            %\vspace{4pt}
            #2
        \end{minipage}
    }
}

\newcommand{\greybox}[2][gray!40]{
    \colornote[#1]{#2}
}

\newcommand{\margingreybox}[2][gray!20]{
    \margincolornote[#1]{#2}
}

\newcommand{\textmath}[1]{
    \textit{#1}
}

\usepackage{csquotes}

\newcommand{\syllogism}[3]{
    \begin{flalign*}
        &\textit{#1}&\\
        &\textit{#2}&\\
        \qquad \qquad \therefore \hphantom{z} &\textit{#3}&
    \end{flalign*}
}

% Quelques classes de théorèmes
\newtheorem{definition}{Définition}
\newtheorem{theorem}{Théorème}
\newtheorem{lemma}{Lemma}[theorem]
\newtheorem{corollary}{Corollaire}[theorem]
\newtheorem*{remark}{\normalfont{\emph{Remarque}}}
\newtheorem{exercise}{\normalfont{\emph{Exercice}}}
\newtheorem*{solution}{\normalfont{\emph{Solution}}}

\usepackage[]{float}

\usepackage{array}
\usepackage{siunitx}

\usepackage{booktabs}

\usepackage[]{basicarith}


% les plots tikzpicture
\usepackage{pgfplots}
\pgfplotsset{compat = newest}
\usetikzlibrary{intersections}

% margin plot
\newenvironment{marplot}[2]{
    \begin{marginfigure}[]
        \caption{\footnotesize #2}
        \begin{tikzpicture}[]
            \begin{axis}[#1]
}{
            \end{axis}
        \end{tikzpicture}
    \end{marginfigure}
}

\newcommand{\colorsquare}[1][red]{
    $\color{#1}\blacksquare$
}

\title{Algèbre}
\author{Tanguy Andreani}

\begin{document}

\maketitle

\begin{abstract}
    Ce document contient des notes prises pour les livres
    \underline{Algebra for College Students} (Lial, Hornsby, McGinnis)
    et \underline{Algebra} (Gelfand).
    
    L'objectif est de comprendre un maximum le fonctionnement
    de l'algèbre et des techniques qui en découlent ainsi
    qu'apprendre à bien formater les démonstration en \LaTeX.
\end{abstract}

\tableofcontents

\section{Introduction}

\subsection{L'algèbre, l'algèbre linéaire, l'algèbre abstrait}



Le mot algèbre désigne plein de choses. Comme d'habitude, la
première étape est de lister les éventuels problèmes de traduction.

\begin{quote}
    L'algèbre est une branche des mathématiques qui permet d'exprimer les
    propriétés des \textit{opérations} et le traitement des \textit{équations}
    et aboutit à l'étude des \textit{structures algébriques}.
\end{quote}

J'imagine qu'en généralisant beaucoup, on parle d'objets indéfinis
\sidenote{\footnotesize J'utilise \textit{indéfini} au sens où on formalise
la notion d'objet, sans faire appelle à l'intuition type "\textit{mon objet
est un nombre, ou un ensemble}", on reste abstrait.}
sur lesquels des opérations sont possibles. D'où les mots comme
\textit{groupe}, \textit{catégorie}, etc. de l'algèbre abstrait.
Ce sont les structures algébriques.

L'algèbre élémentaire est le sujet principal de ces notes.
Là où l'arithmétique (théorie des nombres) se concentre sur les
propriétés de nombre particuliers (les diviseurs par exemple),
l'algèbre élémentaire étudie les opérations sur des nombres "en général".
On étudie par exemple l'addition, la multiplication et
on introduit le concept d'équation.

Bien sûr, les deux sont sûrement interconnectés.

L'objectif est d'avoir une approche \textit{problem-solving}.

\section{Arithmétique basique}

Le livre de Gelfand commence avec un mélange de théorie des
nombres basique et d'algèbre.

\greybox{
    \subsection*{Rappel: l'addition et la multiplication sont commutatives}
    
    Donc on a $a + b = b + a$ et $ab = ba$.
}

\begin{exercise}
    On veut obtenir 1000 en utilisant que des 8 et des $+$.
\end{exercise}

\begin{solution}
    On pose l'addition en laissant vide certaines cases:

    \probline{5}{...8}
    \opline{$+$}{....}
    \opline{$+$}{...8}
    \soluline{1000}
    % \begin{figure}[H]
    %     \centering
    %     \begin{tabular}{r}
    %         $\hphantom{\dots}8$ \\
    %         $+\hphantom{\dots}\hphantom{8}$ \\
    %         $+\hphantom{\dots}8$ \\
    %         \hline
    %         $1000$
    %     \end{tabular}
    % \end{figure}
        
    Vu qu'on veut 0 en premier chiffre du résultat, on doit ajouter
    8 5 fois, de manière à obtenir 40.

    \probline{5}{...8}
    \opline{$+$}{...8}
    \opline{$+$}{...8}
    \opline{$+$}{...8}
    \opline{$+$}{...8}
    \soluline{1000}
    % \begin{figure}[H]
    %     \centering
    %     \begin{tabular}{r}
    %             $\dots 8$ \\
    %             $\dots 8$ \\
    %             $\dots 8$ \\
    %             $\dots 8$ \\
    %             $\dots 8$ \\
    %             \hline
    %             $1000$
    %     \end{tabular}
    % \end{figure}


    On remplie ensuite le tableau avec le même raisonnement vers la gauche.

    \probline{5}{888}
    \opline{$+$}{88}
    \opline{$+$}{8}
    \opline{$+$}{8}
    \opline{$+$}{8}
    \soluline{1000}
    % \begin{figure}[H]
    %     \centering
    %     \begin{tabular}{r}
    %         $888$ \\
    %         $88$ \\
    %         $8$ \\
    %         $8$ \\
    %         $8$ \\
    %         \hline
    %         $1000$
    %     \end{tabular}
    % \end{figure}
\end{solution}

\begin{exercise}
    Trouver tous les termes de l'addition suivante (A, B et C sont distincts):

    \probline{5}{aaa}
    \opline{$+$}{bbb}
    \soluline{aaac}
    % \begin{figure}[H]
    %     \centering
    %     \begin{tabular}{r}
    %         \texttt{AAA} \\
    %         \texttt{BBB} \\
    %         \hline
    %         \texttt{AAAC}
    %     \end{tabular}
    % \end{figure}
\end{exercise}

\begin{solution}
    A, B et C ne peuvent être égaux à 0. (A + B = 0, A + 0 = C, 0 + B = C sont
    trois cas impossibles.)

    On a A = 1 car la somme de deux nombres de trois chiffres
    est forcéments strictement inférieure à 2000. Ici elle est
    aussi supérieure à 1000.

    On a donc

    \probline{5}{111}
    \opline{$+$}{bbb}
    \soluline{111c}

    % \begin{figure}[H]
    %     \centering
    %     \begin{tabular}{r}
    %         \texttt{111} \\
    %         \texttt{BBB} \\
    %         \hline
    %         \texttt{111C}
    %     \end{tabular}
    % \end{figure}

    Comme 1 + B = 10, pour avoir la retenue, B = 9 et C = 0.

    \probline{5}{111}
    \opline{$+$}{999}
    \soluline{1110}
\end{solution}


\section{Systèmes d'équations linéaires}

\subsection{Méthode de substitution}

Avant et pendant l'utilisation de la méthode de substitution,
il peut-être nécessaire de préparer les équations, notemment
en y supprimant les coefficients fractionnaires afin de
simplifier les étapes suivantes.

\begin{flalign*}
    \systeme{
        2x-y=6 @(a),
        x=y+ 2 @(b)
    }
\end{flalign*}

\begin{marginfigure}
    \caption{\footnotesize Représentation du système d'équations}
    \begin{tikzpicture}[]
        \begin{axis}[xmin=0,
            xmax=10,
            ymin=0,
            ymax=10,
            width=6.5cm,
            height=4cm,
            grid=both,
            legend cell align={left},
            legend pos=south east]
            \addplot[blue,name path global=plot1a,domain=0:10, thick] ({x},{2 * x - 6});
            \addplot[red,name path global=plot1b, domain=0:10, thick] ({x},{x - 2});
            \fill
                [name intersections={of=plot1a and plot1b, name=i, total=\t}]
                [orange, opacity=1]
                \foreach \s in {1,...,\t}{(i-\s) circle[radius=2pt]};
        \end{axis}
    \end{tikzpicture}
\end{marginfigure}

\begin{enumerate}[label=\textit{Étape \arabic*.}]
    \item{
        \begin{flalign*}
            2(y + 2) - y = 6
        \end{flalign*}
        Par $(a)$, $(b)$ et la propriété de substitution.}
    \item{
        \begin{align*}
            2(y + 2) - y &= 6\\
            \Leftrightarrow \qquad 2y + 4 - y &= 6 \tag*{propriété distributive}\\
            \Leftrightarrow \qquad \qquad \qquad y &= 2
        \end{align*}
        Par simplification de $(1.)$}
    \item{
        \begin{align*}
            x &= y + 2 \\
            &= 4
        \end{align*}
        Par $(2.)$ et la propriété de substitution.}
    \item{
        $S = {(4, 2)}$ \\
        Par $(2.)$ et $(3.)$}
\end{enumerate}

\subsection{Méthode par élimination}

\begin{flalign*}
    \systeme{
        2x + 3y = -6 @(a),
        4x - 3y = 6 @(b)
    }
\end{flalign*}
\begin{marginfigure}
    \caption{\footnotesize Représentation du système d'équations}
    \begin{tikzpicture}[]
        \begin{axis}[xmin=-3,
            xmax=5,
            ymin=-6,
            ymax=2,
            width=6.5cm,
            height=4cm,
            grid=both,
            legend cell align={left},
            legend pos=south east]
            \addplot[name path=plot2a, blue, domain = -3:5] {-2/3 * x - 2};
            \addplot[name path=plot2b, red, domain = -3:5] {4/3 * x - 2};
            \fill
                [name intersections={of=plot2a and plot2b, name=i, total=\t}]
                [orange, opacity=1]
                \foreach \s in {1,...,\t}{(i-\s) circle[radius=2pt]};
        \end{axis}
    \end{tikzpicture}
\end{marginfigure}

\begin{enumerate}[label=\textit{Étape \arabic*.}]
    \item{
        \begin{align*}
            2x + 3y + (4x - 3y) &= -6 + 6 \tag*{$(a) + (b)$} \\
            \Leftrightarrow \qquad \qquad \qquad \qquad 6x &= 0\\
            \Leftrightarrow \qquad \qquad \qquad \qquad x &= 0
        \end{align*}
        Par $(a)$, $(b)$ et la propriété additive.}
    \item{
        \begin{align*}
            2x + 3y &= -6 \tag*{$(a)$} \\
            \Leftrightarrow \qquad \qquad 3y &= -6 \tag*{$(1)$, propriété de substitution}\\
            \Leftrightarrow \qquad \qquad y &= -2 \\
        \end{align*}}
    \item{ $S = {(0, 2)}$ \\
        Par $(1.)$ et $(2.)$}
\end{enumerate}


\end{document}