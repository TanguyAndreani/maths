\documentclass[fleqn,a4paper,nobib]{tufte-handout}

\usepackage[]{amsmath,amsthm,amssymb}
\usepackage[]{systeme}
% accès à enumerate
\usepackage{enumitem}

%Set the font (output) encoding
%--------------------------------------
\usepackage[T1]{fontenc} %Not needed by LuaLaTeX or XeLaTeX

%French-specific commands
%--------------------------------------
% nécessite fontenc (voir au dessus)
\usepackage[french]{babel}
\usepackage[autolanguage]{numprint} % for the \nombre command

%Hyphenation rules
%--------------------------------------
\usepackage{hyphenat}
\hyphenation{mate-mática recu-perar}

\usepackage{xcolor,graphicx}

\newcommand{\colornote}[2][gray!40]{
    \hphantom{hello}

    %\noindent
    \hspace{-\parindent}
    \hspace{-12pt}
    \colorbox{#1}{
        \begin{minipage}{\textwidth}
            %\vspace{4pt}
            \parindent12pt
            #2
            \vspace{4pt}
        \end{minipage}
    }

    \hphantom{hello}
}

\newcommand{\margincolornote}[2][gray!20]{
    \noindent
    \colorbox{#1}{
        \begin{minipage}{\textwidth}
            %\vspace{4pt}
            #2
        \end{minipage}
    }
}

\newcommand{\greybox}[2][gray!40]{
    \colornote[#1]{#2}
}

\newcommand{\margingreybox}[2][gray!20]{
    \margincolornote[#1]{#2}
}

\newcommand{\textmath}[1]{
    \textit{#1}
}

\usepackage{csquotes}

\newcommand{\syllogism}[3]{
    \begin{flalign*}
        &\textit{#1}&\\
        &\textit{#2}&\\
        \qquad \qquad \therefore \hphantom{z} &\textit{#3}&
    \end{flalign*}
}

% Quelques classes de théorèmes
\newtheorem{definition}{Définition}
\newtheorem{theorem}{Théorème}
\newtheorem{lemma}{Lemma}[theorem]
\newtheorem{corollary}{Corollaire}[theorem]
\newtheorem*{remark}{\normalfont{\emph{Remarque}}}
\newtheorem{exercise}{\normalfont{\emph{Exercice}}}
\newtheorem*{solution}{\normalfont{\emph{Solution}}}

\usepackage[]{float}

\usepackage{array}
\usepackage{siunitx}

\usepackage{booktabs}

\usepackage[]{basicarith}


% les plots tikzpicture
\usepackage{pgfplots}
\pgfplotsset{compat = newest}
\usetikzlibrary{intersections}

% margin plot
\newenvironment{marplot}[2]{
    \begin{marginfigure}[]
        \caption{\footnotesize #2}
        \begin{tikzpicture}[]
            \begin{axis}[#1]
}{
            \end{axis}
        \end{tikzpicture}
    \end{marginfigure}
}

\newcommand{\colorsquare}[1][red]{
    $\color{#1}\blacksquare$
}

\title{Théorie des ensembles}
\author{Tanguy Andreani}
\date{Dernière mise à jour: \today}

\begin{document}

\maketitle

\begin{abstract}
    Plusieurs manières d'aborder la théorie des ensembles 
    coexistent. Je vais me baser sur les livres \underline{Naive
    Set Theory} de Paul R. Halmos et \underline{A Book of Set
    Theory} de Charles C. Pinter.
\end{abstract}

\tableofcontents

\section*{Organisation}

\textbf{Attention. Ce texte est certainement rempli d'erreurs.}

À mon avis, le mieux est d'éviter de trop inclure de
détails historiques dans le texte principal en les écrivant
plutôt dans la marge.

Il faudra aussi utiliser \texttt{\\colorsquare[red]} dès
que c'est nécessaire pour créer des repères visuels plus
faciles à lire.

\section*{État du document}

Pour l'instant il y a juste le tout début du livre de Pinter.
En quoi la théorie de Cantor est majeure. L'idée que la
notion d'ensemble ne soit pas aussi intuitive que ça et
amène à des paradoxes.
Le concept d'ensembles infinis. Les paradoxes autour de la
conception intuitive des ensembles.
Comment l'approche axiomatique ambitionne de régler ces
problèmes.

\section{Introduction}

La théorie des ensembles amène la notion d'ensembles infinis.

\subsection{Deux infinis}

\begin{description}
    \item[L'infini "réel"] où l'on considère une infinité
    d'objets qui existent simultanément. (Paradoxe de Zeno,
    ...)
    \item[L'infini potentiel] soit la possibilité de
    dépasser n'importe quantité finie.
\end{description}

À l'époque de Cantor
\sidenote{
    Georg Cantor (1845-1918), a publié ses premiers
    articles sur la théorie des ensembles de 1873 à 1897.
    Malgré les remouts, la théorie se répand vite car
    elle sert à de nombreuses branches des mathématiques.
},
on traite beaucoup plus volontier de l'infini potentiel.
Pourtant c'est bien de l'infini réel qu'il traite avec
les ensembles.

\subsection{Fondement des mathématiques}


Tout objet mathématique ou presque peut être décrit avec un ensemble et
toutes les connaissances mathématiques peuvent être prouvées à l'intérieur
d'\textit{une} théorie des ensembles.

Il existe plus qu'une seule théorie des ensembles, basées sur
des principes différents et avec des objectifs différents.

Source: \href{https://fr.wikipedia.org/wiki/Fondements_des_math%C3%A9matiques}{Wikipedia/Fondements\_des\_mathématiques}.

\subsection{Paradoxes dans la théorie de Cantor}

Certains paradoxes touchent des notions clés de la théorie de Cantor.
Ils sont divisés en deux catégories: les paradoxes \textit{sémantiques}
et \textit{logiques}.

\subsection*{Exemple de paradoxe logique: paradoxe de Russell}

Soit $S$ l'ensemble de tous les ensembles qui ne sont pas des éléments
d'eux même. Si $S$ est un élément de lui même, alors il se contient lui
même et ne peut pas être dans \(S\)S. Si $S$ n'est pas un élément de lui même,
alors il est un élément de lui même. C'est une contradiction.

Faut-il remettre en question la notion d'ensemble d'ensembles ?
Celle d'un ensemble qui se contient lui même? Ou encore celle
de l'existence de $S$?

\subsection*{Exemple de paradoxe sémantique: paradoxe de Berry}

Soit $T$ l'ensemble de\textit{tous les entiers naturels qui peuvent
être décris en moins de vingt mots.} Soit $n$
le\textit{plus petit entier naturel qui ne peut être décris en moins de vingt mots.}
Cette définition fait quatorze mots. Donc $n$ appartient à $T$ car on peut le décrire
en quatorze mots et $n$ n'appartient pas à $T$ par la définition de $n$, ce qui est
une contradiction.

La encore, on est forcés d'admettre que l'ensemble $T$ n'existe pas,
puisque son existence implique une contradiction.

Comment éviter de créer de tels ensembles? Faut-il restreindre la
formulation de la définition d'un ensemble ? 

Une tentative de répondre aux paradoxes est la \textit{méthode axiomatique}.

\subsection{La méthode axiomatique}

La conception intuitive de Cantor n'est pas une base assez solide
pour la théorie des ensembles (notemment en tant que fondement des
mathématiques).

On a commencé par tenter d'exclure des cas spécifiques
\sidenote{\footnotesize lesquels?} mais il a
vite fallu de nouvelles approches de la notion d'ensemble.

\textbf{L'approche axiomatique} \colorsquare[red]

C'est une approche générale des mathématiques qui remonte aux
\marginpar{\colorsquare[red]\footnotesize
    La méthode axiomatique a une histoire relativement récente et
    une grosse partie de l'approche moderne du raisonnement déductif
    sur la base d'axiomes vient de la géometrie. L'avènement des
    géométries non-euclidienne et l'idée que les axiomes ne sont pas
    des vérités universelles.
}
Éléments d'Euclide.

L'idée est de formaliser au maximum, en évitant de faire appel à
l'intuition pour définir une propriété d'un objet. (ex: Euclide
qui ne définit pas ce que signifie pour un point d'être \textit{entre}
deux autres.)

Pour cela, on redéfinit des termes pourtant issus de l'intuition.

Par exemple, un point n'est plus le concept de point sur un plan
physique qu'on s'imagine mais simplement un objet possédant telle
ou telle propriété (avoir des coordonnées par exemple
\sidenote{C'est juste un exemple un peu au hasard.}).

Ça nécessite de développer un langage très précis et sans
ambiguïté pour définir les objets et les propriétés dont on parle.

Pinter parle des prédicats, par exemple, $A(X, Y)$ pour dire
que \textit{X est parallèle à Y}. $X$ et $Y$ sont les
variables du prédicat. 

Certains prédicats fondamentaux sont \textit{élémentaires}.

Dans nos théories, les variables des prédicats peuvent représenter
un objet, une paire d'objets ou encore un \textit{tuplet}
\sidenote{La notion de tuplet est-elle formalisée? Y a-t-il des
propriétés intéressantes ?} d'objets.

Le seul prédicat élémentaire de la théorie des ensembles
est le prédicat d'\textit{appartenance}, $P(x, S) \equiv x \in S$.
\sidenote{C'est la bonne syntaxe ?}

\subsection{L'approche logique}

\subsection{L'approche intuitioniste}


\end{document}