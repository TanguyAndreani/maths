\documentclass[fleqn,a4paper,nobib]{tufte-handout}

\usepackage[]{amsmath,amsthm,amssymb}
\usepackage[]{systeme}
% accès à enumerate
\usepackage{enumitem}

%Set the font (output) encoding
%--------------------------------------
\usepackage[T1]{fontenc} %Not needed by LuaLaTeX or XeLaTeX

%French-specific commands
%--------------------------------------
% nécessite fontenc (voir au dessus)
\usepackage[french]{babel}
\usepackage[autolanguage]{numprint} % for the \nombre command

%Hyphenation rules
%--------------------------------------
\usepackage{hyphenat}
\hyphenation{mate-mática recu-perar}

\usepackage{xcolor,graphicx}

\newcommand{\colornote}[2][gray!40]{
    \hphantom{hello}

    %\noindent
    \hspace{-\parindent}
    \hspace{-12pt}
    \colorbox{#1}{
        \begin{minipage}{\textwidth}
            %\vspace{4pt}
            \parindent12pt
            #2
            \vspace{4pt}
        \end{minipage}
    }

    \hphantom{hello}
}

\newcommand{\margincolornote}[2][gray!20]{
    \noindent
    \colorbox{#1}{
        \begin{minipage}{\textwidth}
            %\vspace{4pt}
            #2
        \end{minipage}
    }
}

\newcommand{\greybox}[2][gray!40]{
    \colornote[#1]{#2}
}

\newcommand{\margingreybox}[2][gray!20]{
    \margincolornote[#1]{#2}
}

\newcommand{\textmath}[1]{
    \textit{#1}
}

\usepackage{csquotes}

\newcommand{\syllogism}[3]{
    \begin{flalign*}
        &\textit{#1}&\\
        &\textit{#2}&\\
        \qquad \qquad \therefore \hphantom{z} &\textit{#3}&
    \end{flalign*}
}

% Quelques classes de théorèmes
\newtheorem{definition}{Définition}
\newtheorem{theorem}{Théorème}
\newtheorem{lemma}{Lemma}[theorem]
\newtheorem{corollary}{Corollaire}[theorem]
\newtheorem*{remark}{\normalfont{\emph{Remarque}}}
\newtheorem{exercise}{\normalfont{\emph{Exercice}}}
\newtheorem*{solution}{\normalfont{\emph{Solution}}}

\usepackage[]{float}

\usepackage{array}
\usepackage{siunitx}

\usepackage{booktabs}

\usepackage[]{basicarith}


% les plots tikzpicture
\usepackage{pgfplots}
\pgfplotsset{compat = newest}
\usetikzlibrary{intersections}

% margin plot
\newenvironment{marplot}[2]{
    \begin{marginfigure}[]
        \caption{\footnotesize #2}
        \begin{tikzpicture}[]
            \begin{axis}[#1]
}{
            \end{axis}
        \end{tikzpicture}
    \end{marginfigure}
}

\newcommand{\colorsquare}[1][red]{
    $\color{#1}\blacksquare$
}

\title{Calculus}
\author{Tanguy Andreani}

\begin{document}

\maketitle

\begin{abstract}
    Ce documents contient des notes et des solutions d'exercices
    du livre \underline{Calculus} de James Stewart.
    D'autres sources vont probablement se rajouter à ça.
\end{abstract}

\tableofcontents

\section*{État du document\sidenote{paragraphe modifié souvent.}}

Je commence à peine le livre. Première étape, faire les exos
de la partie <<\emph{Diagnostic test}>>. Cette note est en pause
jusqu'à ce que j'ai suffisamment retravaillé les exposants.

\section{Diagnostic test}

\textbf{A} Algebra

1. Évaluer chaque expression.

\begin{enumerate}[label=(\alph{*}),leftmargin=2cm,align=left]
    \item $(-3)^4 = 3^2 \times 3^2 = 81$
    \item $-3^4 = -81$
    \item \(3^{-4} = \frac{1}{3^4} = \frac{1}{81}\)
    \item \(\frac{5^{23}}{5^{21}} = 5^2 = 25\)
    \item $(\frac{2}{3})^{-2}$
    \item \(16^{-3/4} = (16^{-3})^{1/4} = \frac{1}{\sqrt[4]{16^3}}
    = \frac{1}{\sqrt[4]{(4\times4)^3}} = \frac{1}{\sqrt[4]{(4\times4)}^3}
    \linebreak = \frac{1}{2^3} = \frac{1}{8} \)\marginnote{(La (f) était compliquée et je connais pas les propriétés pour les exposants réels.)}
\end{enumerate}

2. Simplifier chaque expression sans exposant négatif.

\begin{enumerate}[label=(\alph{*}),leftmargin=2cm,align=left]
    \item $\sqrt[]{200} - \sqrt{32} = $
\end{enumerate}

\end{document}